\chapter{Prévisionnel}
\label{chap:prev}

La première partie de ce projet, présentée dans ce rapport, vient clôturer la phase  d'analyse. Pour la deuxième partie à venir, nous allons continuer notre travail de recherche en répondant à une des difficultés soulevées : le manque de comparaisons entre les deux approches (voir section \ref{sec:complexite-comparaison}).

\section{Travaux à réaliser}

Notre but est de trouver une base sur laquelle comparer les deux approches afin d'identifier des cas concrets où nous pourrions choisir soit le transfer learning, soit le few-shot learning. L'objectif consiste donc à créer puis exploiter cette base de comparaison. Pour ce faire, on peut comprendre plusieurs besoins :
\begin{itemize}
    \item Récupérer et utiliser les implémentations des divers papiers jugés les plus intéressants. Si cette implémentation n'existe pas, elle doit être réalisée par nous-mêmes.
    \item Mettre au point une méthodologie capable de comparer les approches étudiées.
\end{itemize}

\section{Organisation du travail}


\section{}