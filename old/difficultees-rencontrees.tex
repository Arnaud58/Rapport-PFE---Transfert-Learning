\chapter{Difficultés rencontrées}
\label{chap:difficultes}
Cette partie du rapport est destinée à faire un bilan des obstacles rencontrés pendant que nous dressions notre état de l'art.



%----------------------------------------------------------------
\section{Variété des techniques dans la littérature}
La première piste explorée 


%----------------------------------------------------------------
\section{Complexité dans l'analyse et la comparaison de résultats}
\label{sec:complexite-comparaison}

\subsection*{Comparaison entre les deux familles de détecteurs}


%----------------------------------------------------------------
\section{Littérature récente pour le few-shot learning}
Comme dit précédemment, les premiers papiers parlant de la détection d'objet dans le cadre du few-shot learning sont apparus très récemment (2018 avec \cite{DBLP:journals/corr/abs-1804-08328}). Ceci a pour conséquence qu'il existe peu de littérature sur le sujet. L'avantage de cette situation est qu'il était aisé de dresser un état de l'art pour ces méthodes et de comprendre en profondeur la totalité du sujet.

Cependant, la difficulté résidait ici dans le manque de comparaison et dans l'hétérogénéité des méthodes trouvées. En effet, les modèles de FSOD ont, pour la grande majorité, été mises au point durant cette année 2019. De ce fait, ces papiers ont peu voire aucun liens entre eux. Chacun utilisait une approche très différente des autres. De plus, aucune comparaison de résultats entre ces méthodes n'a pu être relevée. Nous devions donc relever deux défis : trouver un moyen de catégoriser les approches présentées et mettre en commun les résultats du mieux possible pour une comparaison.

La solution pour le problème de catégorisation des méthodes a été de réutiliser les catégories de \cite{DBLP:journals/corr/abs-1904-05046} mises en place pour le problème de classification d'images et de les étendre aux techniques de détection d'objet. En ce qui concerne la comparaison des résultats entre les techniques de FSOD, nous avons, dans cette première partie, simplement comparer les scores de mAP sans se soucier de la nature des données prises en compte.