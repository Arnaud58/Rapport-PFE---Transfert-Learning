\SpecialSection{Introduction}
\label{chap:introduction}
% Pour ne pas avoir le mot `Chapitre' au debut de chaque chapitre. 
\NoChapterHead
La détection d'objets est une des tâches de vision artificielle les plus étudiées par la communauté scientifique dû à l'étendue des applications pratiques qu'elle permet (voitures autonomes, vidéo-surveillance, etc). Des progrès significatifs ont été faits ces dernières années grâce à l'apparition des techniques de deep-learning en vision par ordinateur. De nos jours, les algorithmes de détection à base de réseaux de neurones \cite{Object-detection-survey} sont devenus la référence dans la communauté grâce à la précision qu'ils offrent.

Comme les techniques de deep learning font partie de la catégorie des algorithmes d'apprentissage supervisé, les détecteurs à base de réseaux de neurones ont logiquement besoin d'images annotées manuellement pour fonctionner. Ils ont généralement besoin d'un très grand nombre de ces données annotées, malgré le fait que celles-ci soient coûteuses en terme de moyens \cite{Su2012CrowdsourcingAF}.

Pour répondre à ce problème, plusieurs pistes sont explorées par la communauté. Nous pouvons par exemple citer la génération d'images synthétiques permettant de construire des données annotées sans effort \cite{DBLP:journals/corr/RajpuraHB17, Baimukashev2019DeepLB}. Parmi ces pistes explorées, deux nous intéressent : la piste du "transfer learning" (TL) \cite{deep-tl} et la piste de la "few-shot object detection" (FSOD) \cite{FSL-survey}. La première consiste à "transférer" le savoir d'un réseau de neurones vers un autre pour que ce dernier ne doive plus apprendre à partir de zéro. La piste de la FSOD consiste à rendre capable un détecteur de pratiquer une détection sur base d'un petit nombre fixé d'images annotées.

Dans le cadre de notre Master 2 spécialité "Image Vision Interaction" (IVI) au sein de l'Université de Lille, nous devons travailler sur un projet scientifique (PJS) visant à réaliser un travail de recherche sur un sujet en lien avec les thèmes abordés par la spécialisation. C'est dans cette optique-là que nous avons décidé de participer à un travail de recherche destiné à explorer le transfer learning et la FSOD. Nous travaillons sous la supervision de José MENNESSON et Pierre TIRILLY, au sein de l'équipe FOX \footnote{\href{http://www.cristal.univ-lille.fr/FOX/}{Fouille et indexation de dOcuments compleXes et multimedia}} du laboratoire CRIStAL.

Dans ce rapport, nous discutons de notre recherche bibliographique sur ces deux sujets. Nous commençons par placer les bases théoriques des réseaux de neurones nécessaires à la compréhension des techniques de deep learning qui sont utilisées en vision artificielle. Ensuite, nous enchaînons par la mise en contexte des tâches de vision artificielle discutées pendant ce rapport et de l'impact du deep learning sur ces tâches. Nous présentons, par la suite, les approches de transfer learning puis de la few-shot object detection en synthétisant notre bibliographie réalisée sur l'état de l'art de ces deux approches. Après cela, nous tentons de réaliser une analyse et une critique des résultats publiés par la littérature scientifique. Enfin, nous terminons en décrivant le travail pratique qui va suivre ce rapport de bibliographie.