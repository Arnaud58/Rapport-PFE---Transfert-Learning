\SpecialSection{Abstract}

La détection d'objets via l'utilisation du deep learning permet actuellement d'obtenir les meilleurs résultats en terme de précision. Mais de tels détecteurs ont besoin d'être entraînés à partir d'une très grande quantité d'images annotées alors que cette étape d'annotation de données est très coûteuse en terme de moyens. Cependant, l'entraînement de détecteurs à partir d'une quantité limitée d'images reste un problème ouvert dans le domaine. Pour le résoudre, les approches de transfer learning (TL) et de few-shot object detection (FSOD) font partie des solutions envisagées. 

Dans ce rapport, nous conduisons un état de l'art sur ces deux approches. Nous établissons le contexte d'utilisation et une catégorisation propre pour chacune des deux méthodes. Nous comparons ensuite les résultats des modèles constituant l'état de l'art afin de fournir, pour chaque méthode, une base de discussion concernant ses avantages et inconvénients. Nous voyons, à travers ce document, les limitations actuelles de ces méthodes et tentons de comprendre comment les améliorer. Finalement, nous décrivons le travail pratique que nous mènerons ultérieurement et dont le but est d'apporter une comparaison des algorithmes de détection constituant l'état de l'art du transfer learning et de la few-shot object detection.